\documentclass[a4paper,10pt]{article}
\usepackage[utf8]{inputenc}
\usepackage{tikz}
\usetikzlibrary{matrix,shapes,arrows,positioning,chains}

\begin{document}
\section{Intro}
The following charts show the current progress and general program flow.
Green is done, yellow is partial, and red hasn't been started. The program
flow will likely change a large amount as things stop working and we see
better ways of doing it.

\break

\section{Control Flow}
\tikzstyle{startstop} =
	[rectangle,
	rounded corners,
	minimum width=3cm,
	minimum height=1cm,
	text centered,
	draw=black,
	fill=red!30]

\tikzstyle{unfinished} =
	[rectangle,
	rounded corners,
	minimum width=3cm,
	minimum height=1cm,
	text centered,
	draw=black,
	fill=red!30]

\tikzstyle{partial} =
	[rectangle,
	rounded corners,
	minimum width=3cm,
	minimum height=1cm,
	text centered,
	draw=black,
	fill=yellow!30]

\tikzstyle{invisible} =
	[rectangle,
	minimum width=0cm,
	minimum height=0cm]

\tikzstyle{done} =
	[rectangle,
	rounded corners,
	minimum width=3cm,
	minimum height=1cm,
	text centered,
	draw=black,
	fill=green!30]

\tikzstyle{arrow} =
	[thick,
	->,
	>=stealth]
\tikzstyle{line} =
	[thick,
	-,
	>=stealth]

\begin{tikzpicture}[node distance=2cm]

	\node (start) [done] {Program Begin};
	\node (mpv) [done, below of=start] {Start MPV};
	\node (error) [unfinished, right of=start, xshift=8cm] {Error};
	\node (configP) [done, below of=mpv] {Parse Config};
	\node (construct) [done, below of=configP] {Construct Sources};
	\node (sources) [unfinished, below of=construct] {Do any sources offer a config?};
	\node (s2c) [invisible, right of=sources, xshift=2cm] {};
	\node (assets) [partial, below of=sources] {Get all asset files};
	\node (playlists) [done, below of=assets] {Get higest priority playlist};
	\node (getNext) [done, below of=playlists] {Get next playlist item};
	\node (getAsset) [partial, right of=getNext, xshift=2cm] {Find in assets};
	\node (play) [done, right of=getAsset, xshift=2cm] {Play asset};
	\node (p2g) [invisible, above of=play, yshift=-0.8cm] {};
	\node (listen) [unfinished, below of=getNext] {Run sources update every x seconds};

	\draw [arrow] (start) -- (mpv);
	\draw [arrow] (mpv) -- (configP);
	\draw [arrow] (configP) -- (construct);
	\draw [arrow] (construct) -- (sources);
	\draw [line] (sources) -- node [anchor=south] {Yes} (s2c);
	\draw [arrow] (s2c.west) |- (configP);
	\draw [arrow] (sources) -- (assets);
	\draw [arrow] (assets.east) -| node [anchor=south, xshift=-5.2cm] {If no asset file found} (error);
	\draw [arrow] (assets) -- (playlists);
	\draw [arrow] (playlists) -| node [anchor=south, xshift=-5cm] {If no playlist file found} (error);
	\draw [arrow] (playlists) -- (getNext);
	\draw [arrow] (getNext) -- (listen);
	\draw [arrow] (getNext) -- (getAsset);
	\draw [arrow] (getAsset) -- (play);
	\draw [line] (play) -- (p2g);
	\draw [arrow] (p2g.south) -| (getNext);
\end{tikzpicture}

\section{Messaging}
\begin{tikzpicture}[node distance=2cm]
	\node (mGet) [startstop, below of=listen] {Message recieved from source};
	\node (parse) [unfinished, below of=mGet] {Determine the type of message};
	\node (oneShot) [unfinished, below of=parse] {One shot};
	\node (repeat) [unfinished, right of=oneShot, xshift=2cm] {Repeat};
	\node (urgency) [unfinished, below of=oneShot] {Determine urgency of message};
	\node (urgent) [unfinished, below of=urgency] {Now};
	\node (lazy) [unfinished, right of=urgent, xshift=2cm] {After video};
	\node (lazier) [unfinished, right of=urgent, xshift=6cm] {After playlist};
	\node (content) [unfinished, below of=urgent] {The type of content carried in the message};
	\node (playlists) [unfinished, below of=content] {Replace playlist};
	\node (assets) [unfinished, right of=playlists, xshift=2cm] {Replace assets};

	\draw [arrow] (mGet) -- (parse);
	\draw [arrow] (parse) -- (oneShot);
	\draw [arrow] (parse) -- (repeat);
	\draw [arrow] (oneShot) -- (urgency);
	\draw [arrow] (repeat) -- (urgency);
	\draw [arrow] (urgency) -- (urgent);
	\draw [arrow] (urgency) -- (lazy);
	\draw [arrow] (urgency) -- (lazier);
	\draw [arrow] (urgent) -- (content);
	\draw [arrow] (lazy) -- (content);
	\draw [arrow] (lazier) -- (content);
	\draw [arrow] (content) -- (playlists);
	\draw [arrow] (content) -- (assets);


\end{tikzpicture}
\break
\break
The following is even more likely to change.

A source is only allowed to modify its own assets and playlists. A given
source has two locations where it can store data. The first is the oneshot
location. After a oneshot playlist finishes, it is removed. Assets will
remain. A repeat playlist stays around and is repeated after finishing.
A repeat asset stays around as well.

\end{document}
